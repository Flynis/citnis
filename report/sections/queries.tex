\documentclass{report}

\begin{document}
\chapter{Запросы}

\section{Представления}

Для того чтобы облегчить себе реализацию запросов создадим несколько
представлений. Описанные здесь представления находится в файле 
\textbf{citnisdb-views.sql}.

1. Представление с текущим городом, как аналог глобальной константе.
Очевидно что, текущий город уже должен быть в таблице Cities.

\begin{lstlisting}
CREATE VIEW current_city AS
SELECT *
    FROM cities 
    WHERE city_name = 'Novosibirsk'; 
\end{lstlisting}

2. Представление с полным адресом, потому что часто будем фильтровать
по разным частям адреса.

\begin{lstlisting}
CREATE VIEW full_address AS
SELECT *
    FROM addresses 
    JOIN streets USING(street_id)
    JOIN districts USING(district_id)
    JOIN cities USING(city_id);
\end{lstlisting}

3. Абоненты АТС
\begin{lstlisting}
CREATE VIEW ats_subscribers AS
SELECT *
    FROM ats
    JOIN phone_numbers_v USING(ats_id)
    JOIN subscriptions USING(phone_id)
    JOIN subscribers USING(subscriber_id);
\end{lstlisting}

\section{Реализация запросов}

Запросы можно найти в файле \textbf{citnisdb-queries.sql}.
Переменные в запросе будем обозначать как \$(название-переменной). 
Везде где требуется найти общее число записей, просто заменяем
\textbf{SELECT названия столбцов FROM} на 
\textbf{SELECT COUNT (*) FROM}.

\textbf{1. Получить} перечень и общее число абонентов указанной АТС полностью, 
только льготников, по возрастному признаку, по группе фамилий

\begin{lstlisting}
SELECT first_name, last_name, surname, gender, age 
    FROM $(ats_type_view) 
    JOIN ats_subscribers USING(ats_id)
    WHERE (age >= $(subscriber_age) AND
        benefit >= $(subscriber_benefit));
\end{lstlisting}

\textbf{2. Получить} перечень и общее число свободных телефонных 
номеров на указанной АТС, по всей ГТС, по признаку возможности установки 
телефона в данном районе.

\begin{lstlisting}
SELECT phone_no 
    FROM phone_numbers_v
    LEFT JOIN subscriptions USING(phone_id)
    JOIN full_address USING(address_id)
    JOIN ats USING(ats_id)
    WHERE subscriptions.phone_id IS NULL AND 
        district_name = $(district_name) AND 
        ($(all_ats_query) OR ats_name = $(desired_ats_name));
\end{lstlisting}

\textbf{3. Получить} перечень и общее число должников на указанной АТС, 
по всей ГТС, по данному району, абонентов, которые имеют задолженность 
уже больше недели (месяца), по признаку задолженности за межгород и (или) 
по абонентской плате, по размеру долга.

\begin{lstlisting}
SELECT first_name, last_name, surname, gender, age
    FROM ats_subscribers 
    JOIN full_address USING(address_id)
    JOIN service_connection USING(subscription_id)
    JOIN services USING(service_id)
    WHERE district_name = $(district_name) AND 
        ($(all_ats_query) OR ats_name = $(desired_ats_name)) AND 
        debt >= $(debt) AND 
        service_name = "Intercity call" AND
        EXTRACT(DAY FROM (CURRENT_TIMESTAMP - payment_date)) > 7;
\end{lstlisting}

\textbf{4. Определить} АТС (любого или конкретного типа), на которой 
самое большое (маленькое) число должников, самая большая сумма задолженности.

\begin{lstlisting}
WITH ats_debtors AS (
    SELECT ats_name, COUNT(subscriber_id) AS debtors_count
    FROM $(ats_type_view) 
    JOIN ats_subscribers USING(ats_id)
    WHERE debt > 0
    GROUP BY ats_name
    ), 
    ats_min_debtors AS (
    SELECT MIN(debtors_count)
    FROM ats_debtors 
    )
SELECT ats_name
    FROM ats_debtors
    WHERE debtors_count IN (SELECT * FROM ats_min_debtors);

WITH ats_debt AS (
    SELECT ats_name, SUM(debt) AS total_debt
    FROM $(ats_type_view) 
    JOIN ats_subscribers USING(ats_id)
    GROUP BY ats_name
    ), 
    ats_max_debt AS (
    SELECT MAX(total_debt)
    FROM ats_debt 
    )
SELECT ats_name
    FROM ats_debt
    WHERE total_debt IN (SELECT * FROM ats_max_debt);
\end{lstlisting}

\textbf{5. Получить} перечень и общее число общественных телефонов и 
таксофонов во всем городе, принадлежащих указанной АТС, по признаку 
нахожения в данном районе.

\begin{lstlisting}
SELECT phone_no
    FROM payphones_v
    JOIN ats USING(ats_id)
    JOIN full_address USING(address_id)
    WHERE ($(all_ats_query) OR ats_name = $(desired_ats_name))
        AND district_name = $(district_name);
\end{lstlisting}

\textbf{6. Найти} процентное соотношение обычных и льготных абонентов 
на указанной АТС, по всей ГТС, по данному району, по типам АТС.

\begin{lstlisting}
WITH ats_subscribers_count AS (
    SELECT ats_id, COUNT(subscriber_id) AS subscribers_count  
        FROM ats_subscribers
        GROUP BY ats_name
    ),
    ats_beneficiaries_count AS (
    SELECT ats_id, COUNT(subscriber_id) AS beneficiaries_count  
        FROM ats_subscribers
        WHERE benefit >= 0.5
        GROUP BY ats_name
    )
SELECT ats_name, (beneficiaries_count / subscribers_count * 100) AS beneficiaries_percent
    FROM ats
    JOIN ats_subscribers_count USING(ats_id)
    JOIN ats_beneficiaries_count USING(ats_id)
    WHERE ats_name = $(desired_ats_name);
\end{lstlisting}

\textbf{7. Получить} перечень и общее число абонентов указанной АТС, 
по всей ГТС, по данному району, по типам АТС имеющих параллельные телефоны, 
только льготников имеющих параллельные телефоны.

\begin{lstlisting}
SELECT first_name, last_name, surname, gender, age
    FROM $(ats_type_view) 
    JOIN ats_subscribers USING(ats_id)
    JOIN full_address USING(address_id)
    WHERE district_name = $(district_name) AND
        phone_type = $(phone_type) AND
        benefit >= $(benefit);
\end{lstlisting}

\textbf{8. Определить}, есть ли по данному адресу телефон, общее 
количество телефонов и (или) количество телефонов с выходом на межгород, 
с открытым выходом на межгород в данном доме, на конкретной улице.

\begin{lstlisting}
SELECT phone_no
    FROM phone_numbers_v
    JOIN full_address USING(address_id)
    LEFT JOIN subscriptions USING(phone_id)
    LEFT JOIN service_connection USING(subscription_id)
    WHERE street_name = $(street_name) AND
        house_no = $(house_no) AND
        service_name = 'Intercity call';
\end{lstlisting}

\textbf{9. Определить} город, с которым происходит большее количество 
междугородных переговоров.

\begin{lstlisting}
WITH city_calls_count AS (
    SELECT city_name, COUNT(address_id) AS calls_count
    FROM call_log 
    JOIN phone_numbers_v ON phone_numbers_v.phone_id = call_log.recipient
    JOIN full_address USING(address_id)
    GROUP BY city_name
    ), 
    city_max_calls AS (
    SELECT MAX(calls_count)
    FROM city_calls_count 
    )
SELECT city_name
    FROM city_calls_count
    WHERE calls_count IN (SELECT * FROM city_max_calls);
\end{lstlisting}

\textbf{10. Получить} полную информацию об абонентах с заданным телефонным номером.

\begin{lstlisting}
SELECT first_name, last_name, surname, gender, age
    FROM subscribers
    JOIN subscriptions USING(subscriber_id)
    JOIN phone_numbers_v USING(phone_id)
    WHERE phone_no = $(desired_phone_no);
\end{lstlisting}

\textbf{11. Получить} перечень спаренных телефонов, для которых есть 
техническая возможность заменить их на обычные (выделить дополнительный номер).

\begin{lstlisting}
WITH free_phones_in_house AS (
    SELECT address_id, COUNT(phone_id) AS free_phones_count
        FROM phone_numbers_v
        LEFT JOIN subscriptions USING(phone_id)
        JOIN full_address USING(address_id)
        WHERE subscriptions.phone_id IS NULL
        GROUP BY address_id
    )
SELECT phone_no, house_no, street_name, district_name
    FROM phone_numbers_v
    JOIN full_address USING(address_id)
    JOIN free_phones_in_house USING(address_id)
    WHERE free_phones_count > 0;
\end{lstlisting}

\textbf{12. Получить} перечень и общее число должников на указанной АТС, 
по всей ГТС, по данному району, которым следует послать письменное уведомление, 
отключить телефон и(или) выход на межгород.

\begin{lstlisting}

\end{lstlisting}

\end{document}