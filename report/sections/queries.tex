\documentclass{report}

\begin{document}
\chapter{Запросы}

\section{Представления}

Для того чтобы облегчить себе реализацию запросов создадим несколько
представлений. Описанные здесь представления находятся в файле 
\textbf{citnisdb-views.sql}.

\textbf{1.} Представление с полным адресом, потому что часто будем фильтровать
по разным частям адреса.

\begin{lstlisting}
CREATE VIEW full_address AS
SELECT *
    FROM addresses 
    JOIN streets USING(street_id)
    JOIN districts USING(district_id)
    JOIN cities USING(city_id);
\end{lstlisting}

\textbf{2.} Представление с текущим городом. Содержит все адреса города.

\begin{lstlisting}
CREATE VIEW current_city AS
SELECT *
    FROM full_address 
    WHERE city_name = ' Новосибирск ';
\end{lstlisting}

\textbf{3.} Представление с организациями-владельцами АТС.

\begin{lstlisting}
CREATE VIEW ats_owners AS
SELECT ats_id, serial_no, org_name
    FROM ats 
    JOIN organizations using(org_id);
\end{lstlisting}

\section{Реализация запросов}

Запросы реализуем следующим образом: для каждого запроса
создадим представление (либо используем существующее), 
из которого будем получать данные. Клиентские приложения
будут обращаться \textbf{только} к представлениям. Таким образом у нас 
получится интерфейс базы данных, что позволит нам при необходимости 
изменять реализацию представлений не ломая клиентские 
приложения. 

Скрипт с представлениями можно найти в файле \textbf{citnisdb-query-views.sql},
а примеры запросов в файле \textbf{citnisdb-queries.sql}.

\subsection{Перечень абонентов АТС}

\textbf{Получить} перечень и общее число абонентов указанной АТС полностью, 
только льготников, по возрастному признаку, по группе фамилий

\begin{lstlisting}
CREATE VIEW ats_subscribers AS
    SELECT serial_no, address_id, phone_no, phone_type, 
        last_name, first_name, surname, age, gender, benefit, subscription_id
        FROM ats
        JOIN phone_numbers_v USING(ats_id)
        JOIN subscriptions USING(phone_id)
        JOIN subscribers USING(subscriber_id);
\end{lstlisting}

Пример запроса от клиентского приложения:

\begin{lstlisting}
SELECT last_name, first_name, surname, gender, age, benefit
	FROM ats_subscribers
	WHERE (serial_no = '6IBVG115145') AND
		(age >= 18) AND
		(age <= 100) AND
		(benefit >= 0.5) AND
		(LEFT(last_name, 1) = 'А')
	ORDER BY last_name ASC;
\end{lstlisting}

\subsection{Перечень свободных телефонных номеров}

\textbf{Получить} перечень и общее число свободных телефонных 
номеров на указанной АТС, по всей ГТС, по признаку возможности установки 
телефона в данном районе.

\begin{lstlisting}
CREATE VIEW free_phone_numbers AS
    SELECT serial_no, phone_id, phone_no, address_id, district_name, street_name, house_no
        FROM phone_numbers_v
        LEFT JOIN subscriptions USING(phone_id)
        JOIN current_city USING(address_id)
        JOIN ats USING(ats_id)
        WHERE subscriptions.phone_id IS NULL;
\end{lstlisting}

Пример запроса от клиентского приложения:

\begin{lstlisting}
SELECT phone_no, street_name, house_no
	FROM free_phone_numbers
	WHERE (serial_no = '7IBFA115116') AND
		(district_name = ' Дзержинский ')
	ORDER BY phone_no ASC;
\end{lstlisting}

\subsection{Перечень должников}

\textbf{Получить} перечень и общее число должников на указанной АТС, 
по всей ГТС, по данному району, абонентов, которые имеют задолженность 
уже больше недели (месяца), по признаку задолженности за межгород и (или) 
по абонентской плате, по размеру долга.

\begin{lstlisting}
CREATE VIEW debtors AS
    SELECT serial_no, address_id, last_name, first_name, surname, service_name, district_name, street_name, house_no,
            date_diff_in_days(debt_formation_date, payment_date) AS debt_duration_in_days,
            date_diff_in_months(debt_formation_date, payment_date, 20) * service_cost AS debt_amount
        FROM ats_subscribers
        JOIN service_connection USING(subscription_id)
        JOIN current_city USING(address_id)
        JOIN services USING(service_id),
        debt_formation_date()
        WHERE payment_date < debt_formation_date;
\end{lstlisting}

Пример запроса от клиентского приложения:

\begin{lstlisting}
SELECT last_name, first_name, debt_amount
	FROM debtors
	WHERE (serial_no = '6IBVG115145') AND
		(district_name = ' Дзержинский ') AND
		(debt_duration_in_days >= 31) AND
		(service_name = ' Междугородний звонок ')
	ORDER BY last_name ASC;
\end{lstlisting}

\subsection{Статистика по должникам}

\textbf{Определить} АТС (любого или конкретного типа), на которой 
самое большое (маленькое) число должников, самая большая сумма задолженности.

\begin{lstlisting}
CREATE VIEW ats_debt_stat AS 
    SELECT serial_no, COUNT(*) AS debtors_count, SUM(debt_amount) AS total_debt 
        FROM debtors
        GROUP BY serial_no;  
\end{lstlisting}

Примеры запросов от клиентского приложения:

АТС с наибольшей задолжностью

\begin{lstlisting}
SELECT serial_no, MAX(total_debt) AS max_total_debt
	FROM ats_debt_stat
	GROUP BY serial_no;
\end{lstlisting}

АТС с наибольшим числом должников

\begin{lstlisting}
SELECT serial_no, MAX(debtors_count) AS max_debtors_count
	FROM ats_debt_stat
	GROUP BY serial_no;
\end{lstlisting}

АТС с наименьшим числом должников

\begin{lstlisting}
SELECT serial_no, MIN(debtors_count) AS min_debtors_count
	FROM ats_debt_stat
	GROUP BY serial_no;
\end{lstlisting}

\subsection{Перечень таксофонов}

\textbf{Получить} перечень и общее число общественных телефонов и 
таксофонов во всем городе, принадлежащих указанной АТС, по признаку 
нахожения в данном районе.

\begin{lstlisting}
CREATE VIEW payphones_by_ats AS
    SELECT serial_no, phone_id, phone_no, address_id, district_name, street_name, house_no
        FROM payphones_v
        JOIN current_city USING(address_id)
        JOIN city_ats_v USING(ats_id);
\end{lstlisting}

Пример запроса от клиентского приложения:

\begin{lstlisting}
SELECT phone_no, street_name, house_no
	FROM payphones_by_ats
	WHERE (serial_no = '7IBFA115116') AND
		(district_name = ' Железнодорожный ')
	ORDER BY phone_no ASC;
\end{lstlisting}

\subsection{Соотношение обычных и льготных абонентов}

\textbf{Найти} процентное соотношение обычных и льготных абонентов 
на указанной АТС, по всей ГТС, по данному району, по типам АТС.

\begin{lstlisting}
CREATE VIEW ats_beneficiaries_count_by_district AS 
    SELECT serial_no,  
        district_name,  
        COUNT(*) FILTER (WHERE benefit >= 0.5) AS beneficiaries_count, 
        COUNT(*) AS total_subscribers
        FROM ats_subscribers
        JOIN current_city USING(address_id)
        GROUP BY serial_no, district_name;
\end{lstlisting}

Примеры запросов от клиентского приложения:

Соотношение на указанной АТС
\begin{lstlisting}
SELECT SUM(beneficiaries_count) AS ben_cnt, SUM(total_subscribers) AS total_subs
	FROM ats_beneficiaries_count_by_district
	WHERE (serial_no = '6IBVG115145');
\end{lstlisting}

Соотношение по указанному району
\begin{lstlisting}
SELECT SUM(beneficiaries_count) AS ben_cnt, SUM(total_subscribers) AS total_subs
	FROM ats_beneficiaries_count_by_district
	WHERE (district_name = ' Дзержинский ')
\end{lstlisting}

\subsection{Перечень владельцев параллельных телефонов}

\textbf{Получить} перечень и общее число абонентов указанной АТС, 
по всей ГТС, по данному району, по типам АТС имеющих параллельные телефоны, 
только льготников имеющих параллельные телефоны.

\begin{lstlisting}
CREATE VIEW parallel_phone_owners AS
    SELECT serial_no, phone_no, address_id, last_name, first_name, surname, district_name, street_name, house_no
        FROM ats_subscribers
        JOIN current_city USING(address_id)
        WHERE phone_type = 'Parallel';
\end{lstlisting}

Пример запроса от клиентского приложения:

\begin{lstlisting}
SELECT last_name, first_name, phone_no
	FROM parallel_phone_owners
	WHERE (serial_no = '6IBVG115145') AND
		(district_name = ' Дзержинский ') AND
		(benefit >= 0.5)
	ORDER BY last_name ASC;
\end{lstlisting}

\subsection{Найти телефоны по адресу}

\textbf{Определить}, есть ли по данному адресу телефон, общее 
количество телефонов и (или) количество телефонов с выходом на межгород, 
с открытым выходом на межгород в данном доме, на конкретной улице.

\begin{lstlisting}
CREATE VIEW intercity_phone_numbers AS
    SELECT phone_id, address_id, phone_no, ats_id, phone_type
        FROM phone_numbers_v
        JOIN subscriptions USING(phone_id)
        JOIN service_connection USING(subscription_id)
        JOIN services USING(service_id)
        where service_name = 'Междугородний звонок';
\end{lstlisting}

Примеры запросов от клиентского приложения:

Для телефонов с возможностью межгорода
\begin{lstlisting}
SELECT phone_no, street_name, house_no
	FROM intercity_phone_numbers
	JOIN current_city USING(address_id)
	WHERE (street_name = ' Аэропорт ') AND
		(house_no = 35)
	ORDER BY phone_no ASC;
\end{lstlisting}

Для любого телефона
\begin{lstlisting}
    SELECT phone_no, street_name, house_no
	FROM phone_numbers_v
	JOIN current_city USING(address_id)
	WHERE (street_name = ' Аэропорт ') AND
		(house_no = 35)
	ORDER BY phone_no ASC;
\end{lstlisting}

\subsection{Популярный город для междугородних звонков}

\textbf{Определить} город, с которым происходит большее количество 
междугородных переговоров.

\begin{lstlisting}
CREATE VIEW intercity_calls_count AS
    SELECT city_name, COUNT(*) AS calls_count 
        FROM call_log 
        JOIN full_address ON full_address.address_id = call_log.recipient_ats_address
        WHERE city_name NOT IN(SELECT city_name FROM current_city GROUP BY city_name)
        GROUP BY city_name;
\end{lstlisting}

Пример запроса от клиентского приложения:

\begin{lstlisting}
SELECT city_name, MAX(calls_count) AS max_calls_count
	FROM intercity_calls_count
	GROUP BY city_name;
\end{lstlisting}

\subsection{Информация об абоненте по номеру телефона}

\textbf{Получить} полную информацию об абонентах с заданным телефонным номером.

Пример запроса от клиентского приложения:

\begin{lstlisting}
SELECT last_name, first_name
	FROM ats_subscribers
	WHERE (phone_no = 101)
	ORDER BY last_name ASC;
\end{lstlisting}

\subsection{Спаренные телефоны под замену}

\textbf{Получить} перечень спаренных телефонов, для которых есть 
техническая возможность заменить их на обычные (выделить дополнительный номер).

\begin{lstlisting}
CREATE VIEW free_phones_in_house AS 
    SELECT address_id, COUNT(phone_id) AS free_phones_count
        FROM free_phone_numbers
        GROUP BY address_id;
CREATE VIEW paired_phones_for_replacement AS 
    SELECT phone_id, phone_no, address_id, district_name, street_name, house_no
        FROM phone_numbers_v
        JOIN current_city USING(address_id)
        JOIN free_phones_in_house USING(address_id)
        WHERE phone_type = 'Paired' AND free_phones_count > 0;
\end{lstlisting}

Пример запроса от клиентского приложения:

\begin{lstlisting}
SELECT phone_no, street_name, house_no
	FROM paired_phones_for_replacement
	ORDER BY phone_no ASC;
\end{lstlisting}

\subsection{Должники, которым необходимо отправить уведомление}

\textbf{12. Получить} перечень и общее число должников на указанной АТС, 
по всей ГТС, по данному району, которым следует послать письменное уведомление, 
отключить телефон и(или) выход на межгород.

Пример запроса от клиентского приложения:

\begin{lstlisting}
SELECT last_name, first_name, debt_amount
	FROM debtors
	WHERE (serial_no = '6IBVG115145') AND
		(district_name = 'Дзержинский') AND
		(debt_duration_in_days = 2) AND
		(service_name = 'Междугородний звонок')
	ORDER BY last_name ASC;
\end{lstlisting}

\end{document}