\documentclass{report}

\begin{document}
\chapter{Хранимые процедуры и триггеры}

Все описанные в этой главе триггеры находятся в файле 
\textbf{citnisdb-functions.sql}.

\section{Наследование}

Отношение супертип-подтип в нашей схеме выражено как связь 1:1. 
Для того, чтобы нам было удобно работать с подтипами, создадим
для каждого свое представление и все операции INSERT, UPDATE, DELETE
реализуем через триггеры INSTEAD OF на эти представления, которые,
например в случае INSERT, будут осуществлять вставку сначала в 
таблицу супертипа, а потом с тем же ключом в таблицу подтипа.
И для автоматической генерации ключа заведем последовательность 
для каждого базового типа, чтобы ключи у подтипов не повторялись.
Пример для INSERT в City ATS:
\begin{lstlisting}
CREATE SEQUENCE ats_id_seq AS integer;

CREATE VIEW city_ats_v AS
    SELECT *
    FROM ats 
    JOIN city_ats USING(ats_id); 

CREATE FUNCTION city_ats_insert_row() RETURNS trigger AS $$
DECLARE
    id integer := nextval('ats_id_seq');
BEGIN
    INSERT INTO ats (ats_id, ats_owner, 
                    first_phone_no, 
                    last_phone_no)
        VALUES(id, NEW.ats_owner, 
                NEW.first_phone_no, 
                NEW.last_phone_no);
    INSERT INTO city_ats (ats_id)
        VALUES(id);
    RETURN NEW;
END;
$$ LANGUAGE plpgsql;
CREATE TRIGGER city_ats_insert 
    INSTEAD OF INSERT ON city_ats_v
    FOR EACH ROW EXECUTE PROCEDURE city_ats_insert_row();
\end{lstlisting}

Аналогичные триггеры заведем для операций UPDATE и DELETE. 
Повторим это для Department ATS и Institution ATS, 
Payphone и Phone number.

\section{Триггеры}

\textbf{1.} У АТС имеется диапозон доступных номеров телефонов. 
При расширении этого диапозона проблем нет, а вот при его 
уменьшении существующий номер телефона может оказаться 
вне диапозона. Поэтому запретим уменьшать диапозон номеров.

\begin{lstlisting}
CREATE FUNCTION prohibit_ats_reduction_row() RETURNS trigger AS $$
BEGIN
    IF (NEW.first_phone_no > OLD.first_phone_no OR
         NEW.last_phone_no < OLD.last_phone_no)
    THEN RETURN NULL;
    ELSE RETURN NEW;
    END IF;
END;
$$ LANGUAGE plpgsql;
CREATE TRIGGER prohibit_ats_reduction 
    BEFORE UPDATE ON ats
    FOR EACH ROW EXECUTE PROCEDURE prohibit_ats_reduction_row();
\end{lstlisting}

\textbf{2.} При вставке/изменении телефона мы должны проверить, 
что номер телефона попадает в допустимый у АТС диапозон.

\begin{lstlisting}
CREATE FUNCTION check_phone_number_row() RETURNS trigger AS $$
DECLARE
    rec record;
BEGIN
    SELECT * INTO rec FROM ats WHERE ats_id = NEW.ats_id;
    IF (NEW.phone_no > rec.last_phone_no OR
         NEW.phone_no < rec.first_phone_no)
    THEN RETURN NULL;
    ELSE RETURN NEW;
    END IF;
END;
$$ LANGUAGE plpgsql;
CREATE TRIGGER check_phone_number 
    BEFORE INSERT OR UPDATE ON ats
    FOR EACH ROW EXECUTE PROCEDURE check_phone_number_row();
\end{lstlisting}

\textbf{3.} При изменении типа конечного устройства у номера телефона, 
необходимо проверить, что им владеет не больше одного абонента.

\begin{lstlisting}
CREATE FUNCTION phone_type_update_row() RETURNS trigger AS $$
DECLARE
    rec record;
BEGIN
    SELECT COUNT(*) AS owners_count INTO rec 
        FROM subscriptions
        WHERE phone_id = NEW.phone_id;
    IF (rec.owners_count > 1)
    THEN
        RETURN NULL;
    ELSE
        RETURN NEW;
    END IF;
END;
$$ LANGUAGE plpgsql;
CREATE TRIGGER phone_type_update 
    BEFORE UPDATE ON phones
    FOR EACH ROW 
    EXECUTE PROCEDURE phone_type_update_row();
\end{lstlisting}

\textbf{4.} При вставке/обновлении абонемента нам нужно проверить, 
чтобы у обычных и спаренных телефонов был один владелец, 
а у владельцев параллельных - был одинаковый адрес. 
Кроме того, для нового абонемента
будем сразу подключать услугу ``Звонок''.

\begin{lstlisting}
CREATE FUNCTION subscription_phone_check_row() RETURNS trigger AS $$
DECLARE
    rec record;
    ph record;
BEGIN
    SELECT * INTO ph 
        FROM phone_numbers_v 
        WHERE phone_id = NEW.phone_id;
    IF (ph.phone_type = 'Parallel')
    THEN
        SELECT COUNT(*) AS other_address_count INTO rec 
            FROM subscriptions 
            JOIN phones USING(phone_id)
            WHERE (phone_id = ph.phone_id AND 
                    address_id != ph.address_id);
        IF (rec.other_address_count > 0)
        THEN 
            RETURN NULL;
        ELSE 
            RETURN NEW;
        END IF;
    ELSE
        SELECT COUNT(*) AS owners_count INTO rec 
            FROM subscriptions 
            WHERE phone_id = NEW.phone_id;
        IF ((TG_OP = 'INSERT' AND rec.owners_count > 0) OR 
            (TG_OP = 'UPDATE' AND rec.owners_count > 1))
        THEN 
            RETURN NULL;
        ELSE 
            RETURN NEW;
        END IF;
    END IF;
END;
$$ LANGUAGE plpgsql;
CREATE TRIGGER subscription_phone_check 
    BEFORE INSERT OR UPDATE ON subscriptions
    FOR EACH ROW 
    EXECUTE PROCEDURE subscription_phone_check_row();

CREATE FUNCTION subscription_default_service_row() RETURNS trigger AS $$
BEGIN
    INSERT INTO service_connection (subscription_id, service_id)
        VALUES(NEW.subscription_id, 1);
    RETURN NEW;
END;
$$ LANGUAGE plpgsql;
CREATE TRIGGER subscription_default_service
    AFTER INSERT ON subscriptions
    FOR EACH ROW 
    EXECUTE PROCEDURE subscription_default_service_row();
\end{lstlisting}

\section{Хранимые процедуры}

Для отслеживания долга абонентов нам необходимо
работать с датами, поэтому заведем несколько 
полезных функций.

\textbf{1.\ date\_diff\_in\_days} вычиляет разницу между 
датами в днях. 

\begin{lstlisting} 
CREATE FUNCTION date_diff_in_days(date1 date, date2 date) RETURNS int AS $$
BEGIN
    RETURN date_part('day', date1::timestamp - date2::timestamp)::int;
END;
$$ LANGUAGE plpgsql;
\end{lstlisting}

\textbf{2.\ date\_diff\_in\_months} вычиляет разницу между 
датами в месяцах начиная с дня \textbf{start\_day}. 

\begin{lstlisting}
CREATE FUNCTION date_diff_in_months(date1 date, date2 date, start_day int) RETURNS int AS $$
DECLARE
    years int;
    months int;
BEGIN
    years = (date_part('year', date1) - date_part('year', date2))::int;
    months = (date_part('month', date1) - date_part('month', date2))::int;
    RETURN years * 12 + (months + 12) % 12 + (date_part('day', date2) != start_day)::int;
END;
$$ LANGUAGE plpgsql;
\end{lstlisting}

\textbf{3.\ date\_with\_previous\_month} возвращает день \textbf{new\_date\_day} 
прощлого месяца. 

\begin{lstlisting}
CREATE FUNCTION date_with_previous_month(d date, new_date_day int) RETURNS date AS $$
BEGIN
    RETURN make_date(
          date_part('year', make_date(date_part('year', d)::int, date_part('month', d)::int, 1) - interval '1 day')::int,
          date_part('month', make_date(date_part('year', d)::int, date_part('month', d)::int, 1) - interval '1 day')::int,
          new_date_day);
END;
$$ LANGUAGE plpgsql;
\end{lstlisting}

\textbf{4.\ debt\_formation\_date} возвращает дату, с которой
отсчитывается текущий долг. 

\begin{lstlisting}
CREATE FUNCTION debt_formation_date() RETURNS date AS $$
DECLARE
    write_off_date date;
BEGIN
    IF(date_part('day', current_date)::int >= 20)
    THEN write_off_date = make_date(date_part('year', current_date)::int, date_part('month', current_date)::int, 20);
    ELSE write_off_date = date_with_previous_month(current_date, 20);
    END IF;
    RETURN date_with_previous_month(write_off_date, 20);
END;
$$ LANGUAGE plpgsql;
\end{lstlisting}

\end{document}