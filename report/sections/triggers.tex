\documentclass{report}

\begin{document}
\chapter{Хранимые процедуры и триггеры}

Все описанные в этой главе триггеры находится в файле 
\textbf{citnisdb-triggers.sql}.

\section{Наследование}

Отношение супертип-подтип в нашей схеме выражено как связь 1:1. 
Для того, чтобы нам было удобно работать с подтипами, создадим
для каждого свое представление и все операции INSERT, UPDATE, DELETE
реализуем через триггеры INSTEAD OF на эти представления, которые,
например в случае INSERT, будут осуществлять вставку сначала в 
таблицу супертипа, а потом с тем же ключом в таблицу подтипа.
И для автоматической генерации ключа заведем последовательность 
для каждого базового типа, чтобы ключи у подтипов не повторялись.
Пример для INSERT в City ATS:
\begin{lstlisting}
CREATE SEQUENCE ats_id_seq AS integer;

CREATE VIEW city_ats_v AS
    SELECT *
    FROM ats 
    JOIN city_ats USING(ats_id); 

CREATE FUNCTION city_ats_insert_row() RETURNS trigger AS $$
DECLARE
    id integer := nextval('ats_id_seq');
BEGIN
    INSERT INTO ats (ats_id, ats_owner, 
                    first_phone_no, 
                    last_phone_no)
        VALUES(id, NEW.ats_owner, 
                NEW.first_phone_no, 
                NEW.last_phone_no);
    INSERT INTO city_ats (ats_id)
        VALUES(id);
    RETURN NEW;
END;
$$ LANGUAGE plpgsql;
CREATE TRIGGER city_ats_insert 
    INSTEAD OF INSERT ON city_ats_v
    FOR EACH ROW EXECUTE PROCEDURE city_ats_insert_row();
\end{lstlisting}

Аналогичные триггеры заведем для операций UPDATE и DELETE. 
Повторим это для Department ATS и Institution ATS, 
Payphone и Phone number.

\section{Триггеры}

У АТС имеется диапозон доступных номеров телефонов. При расширении этого 
диапозона проблем нет, а вот при его уменьшении существующий номер 
телефона может оказаться вне диапозона. Поэтому запретим уменьшать 
диапозон номеров.

\begin{lstlisting}
CREATE FUNCTION prohibit_ats_reduction_row() RETURNS trigger AS $$
BEGIN
    IF (NEW.first_phone_no > OLD.first_phone_no OR
         NEW.last_phone_no < OLD.last_phone_no)
    THEN RETURN NULL;
    ELSE RETURN NEW;
    END IF;
END;
$$ LANGUAGE plpgsql;
CREATE TRIGGER prohibit_ats_reduction 
    BEFORE UPDATE ON ats
    FOR EACH ROW EXECUTE PROCEDURE prohibit_ats_reduction_row();
\end{lstlisting}

При вставке/изменении телефона мы должны проверить, 
что номер телефона попадает в допустимый у АТС диапозон.

\begin{lstlisting}
CREATE FUNCTION check_phone_number_row() RETURNS trigger AS $$
DECLARE
    rec record;
BEGIN
    SELECT * INTO rec FROM ats WHERE ats_id = NEW.ats_id;
    IF (NEW.phone_no > rec.last_phone_no OR
         NEW.phone_no < rec.first_phone_no)
    THEN RETURN NULL;
    ELSE RETURN NEW;
    END IF;
END;
$$ LANGUAGE plpgsql;
CREATE TRIGGER check_phone_number 
    BEFORE INSERT OR UPDATE ON ats
    FOR EACH ROW EXECUTE PROCEDURE check_phone_number_row();
\end{lstlisting}

При изменении типа конечного устройства у номера телефона, необходимо проверить,
что им владеет не больше одного абонента.

\begin{lstlisting}
CREATE FUNCTION phone_type_update_row() RETURNS trigger AS $$
DECLARE
    rec record;
BEGIN
    SELECT COUNT(*) AS owners_count INTO rec 
        FROM subscriptions
        WHERE phone_id = NEW.phone_id;
    IF (rec.owners_count > 1)
    THEN
        RETURN NULL;
    ELSE
        RETURN NEW;
    END IF;
END;
$$ LANGUAGE plpgsql;
CREATE TRIGGER phone_type_update 
    BEFORE UPDATE ON phones
    FOR EACH ROW 
    EXECUTE PROCEDURE phone_type_update_row();
\end{lstlisting}

Чтобы не терять должников, не будем удалять их. То есть запретим
удаление абонентов с задолжностью.

\begin{lstlisting}
CREATE FUNCTION prohibit_debtor_remove_row() RETURNS trigger AS $$
BEGIN
    IF (OLD.debt > 0)
    THEN RETURN NULL;
    ELSE RETURN OLD;
    END IF;
END;
$$ LANGUAGE plpgsql;
CREATE TRIGGER prohibit_debtor_remove 
    BEFORE DELETE ON subscribers
    FOR EACH ROW 
    EXECUTE PROCEDURE prohibit_debtor_remove_row();
\end{lstlisting}

При вставке/обновлении абонемента нам нужно проверить, чтобы у
обычных и спаренных телефонов был один владелец, а у владельцев
параллельных - был одинаковый адрес. Кроме того, для нового абонемента
будем сразу подключать услугу ``Звонок''.

\begin{lstlisting}
CREATE FUNCTION subscription_phone_check_row() RETURNS trigger AS $$
DECLARE
    rec record;
BEGIN
    IF (NEW.phone_type = 'Parallel')
    THEN
        SELECT COUNT(*) AS other_address_count INTO rec 
            FROM subscriptions 
            JOIN phones USING(phone_id)
            WHERE (phone_id = NEW.phone_id AND 
                    address_id != NEW.address_id);
        IF (rec.other_address_count > 0)
        THEN RETURN NULL;
        ELSE RETURN NEW;
        END IF;
    ELSE
        SELECT COUNT(*) AS owners_count INTO rec 
            FROM subscriptions 
            WHERE phone_id = NEW.phone_id;
        IF ((TG_OP = 'INSERT' AND rec.owners_count > 0) OR 
            (TG_OP = 'UPDATE' AND rec.owners_count > 1))
        THEN RETURN NULL;
        ELSE RETURN NEW;
        END IF;
    END IF;
END;
$$ LANGUAGE plpgsql;
CREATE TRIGGER subscription_phone_check 
    BEFORE INSERT OR UPDATE ON subscriptions
    FOR EACH ROW 
    EXECUTE PROCEDURE subscription_phone_check_row();

CREATE FUNCTION subscription_default_service_row() RETURNS trigger AS $$
BEGIN
    INSERT INTO service_connection (subscription_id, service_id)
        VALUES(NEW.subscription_id, 1);
END;
$$ LANGUAGE plpgsql;
CREATE TRIGGER subscription_default_service
    AFTER INSERT ON subscriptions
    FOR EACH ROW 
    EXECUTE PROCEDURE subscription_default_service_row();
\end{lstlisting}

\end{document}