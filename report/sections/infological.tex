\documentclass{report}

\begin{document}
\chapter{Инфологическое проектирование модели базы данных}

\section{Описание сущностей предметной области}
Перечислим сущности, которые были выявлены в ходе анализа предметной области.

\textbf{Автоматическая телефонная станция (АТС)}. У АТС имеется 
организация-владелец, первый и последний доступные номера телефонов, 
т.е. диапозон доступных номеров телефонов. 
Все эти атрибуты являются описательными, поэтому добавим идентификатор АТС. 
\newline\textbf{АТС}:
\begin{itemize}
    \item идентификатор
    \item владелец (описательный)
    \item первый телефон (описательный)
    \item последний телефон (описательный)
\end{itemize}
Сущность АТС является супертипом для городской, 
ведомственной и учрежденческой АТС, которые обладают характерным только 
для этой группы, пока неизвестным, набором атрибутов.

\textbf{Абонент}. У каждого абонента есть ФИО, 
и следующие описательные атрибуты: пол, возраст, льгота - для подсчета 
абонентской платы, задолжность - для отслеживания долга абонента. 
Также, чтобы избежать случая совпадения ФИО, нам необходим идетификатор.
\newline\textbf{Абонент}:
\begin{itemize}
    \item идентификатор
    \item ФИО (указывающий)
    \item пол (описательный)
    \item возраст (описательный)
    \item льгота (описательный)
    \item задолжность (описательный)
\end{itemize}

\textbf{Номер телефона}. Каждый номер должен принадлежать какой-то АТС, 
чтобы избежать коллизий. Поэтому у номера телефона есть сам номер
и АТС, которой он принадлежит. Кроме того, у телефона есть адрес: 
район, улица, дом. Чтобы первичный ключ состоял из одного атрибута, 
добавим идентификатор. 
\newline\textbf{Номер телефона}:
\begin{itemize}
    \item идентификатор
    \item номер (указывающий)
    \item АТС (вспомогательный)
    \item район (описателный)
    \item улица (описателный)
    \item дом (описателный)
\end{itemize}
Сущность номер телефона является супертипом для обычного номера, 
которым будет пользоваться абонент, и таксофона, который не принадлежит
никакому абоненту. У обычного номера телефона имеется номер квартиры, 
в которой он находится. 

\textbf{Абонемент}. Абонемент закрепляет обычный номер телефона за абонентом.
Также, добавим идентификатор для объединения первичного ключа в один атрибут.
Кроме того, абонемент описывается следующими атрибутами: тип конечного 
устройства (основной, параллельный, спареный), статус 
(выключен, оплачен, не оплачен) и дата оплаты последней оплаты 
(чтобы фиксировать задолжность). 
\newline\textbf{Абонемент}:
\begin{itemize}
    \item идентификатор
    \item обычный номер телефона (вспомогательный)
    \item абонент (вспомогательный)
    \item тип телефона (описателный)
    \item статус (описателный)
    \item дата оплаты (описателный)
\end{itemize}
Сущность абонемент является супертипом для обычного абонемента и междугороднего.
Междугородний абонемент имеет дополнительные описательные атрибуты: 
статус (как у обычного) и дату оплаты, для подсчета абонентской 
платы и задолжности.

\textbf{Междугородний звонок}. Эта сущность необходима для сбора сведений 
о междугородных переговорах. Междугородний звонок может совершать только 
междугородний абонемент. Кроме того, нам необходима информация о звонке такая,
как дата и время - для идентификации, длительность и пункт назначения (город).
\newline\textbf{Междугородний звонок}:
\begin{itemize}
    \item междугородний абонемент (вспомогательный)
    \item дата и время (указывающий)
    \item длительность (описателный)
    \item пункт назначения (описателный)
\end{itemize}

Кроме вышеописанных сущностей нам понадобится сущности описывающие карту 
нашего города и информацию о других городах. Кратко опишем их.
\begin{itemize}
    \item[] \textbf{Город}
    \begin{itemize}
        \item идентификатор
        \item название (указывающий)
    \end{itemize}
    \item[] \textbf{Район}
    \begin{itemize}
        \item идентификатор
        \item город (вспомогательный)
        \item название (указывающий)
    \end{itemize} 
    \item[] \textbf{Улица}
    \begin{itemize}
        \item идентификатор
        \item район (вспомогательный)
        \item название (указывающий)
    \end{itemize} 
\end{itemize}

\section{Описание связей}
\begin{longtblr}[caption = {Таблица связей}]{
        colspec={|X|X[4]|X|X[8]|}, row{1} = {c}, hlines,
    }
    Связь & Сущности & Тип & Описание \\
    R1 & АТС - ведомственная АТС & 1:1 & Связь супертип-подтип \\
    R2 & АТС - учрежденческая АТС & 1:1 & Связь супертип-подтип \\
    R3 & АТС - городская АТС & 1:1 & Связь супертип-подтип \\
    R4 & АТС - номер телефона & 1:M & 
        К АТС подключено множество номеров. У каждого номера только одна АТС \\
    R5 & Номер телефона - обычный номер телефона & 1:1 & 
        Связь супертип-подтип \\
    R6 & Номер телефона - таксофон & 1:1 & 
        Связь супертип-подтип \\
    R7 & Абонент - абонемент & 1:M & 
        У абонента может быть несколько абонементов (номеров телефонов).
        Абонемент имеет только одного абонента \\
    R8 & Обычный номер телефона - абонемент & 1:M & 
        Один и тот же номер может быть в нескольких абонементах, 
        если у абонента параллельный телефон, иначе только в одном 
        абонементе. Абонемент имеет только один номер \\
    R9 & Абонемент - обычный абонемент & 1:1 & 
        Связь супертип-подтип \\
    R10 & Абонемент - междугородний абонемент & 1:1 & 
        Связь супертип-подтип \\
    R11 & Междугородний абонемент - междугородний звонок & 1:M &
        Абонент может совершить несколько звонков. У каждого звонка 
        один инициатор \\
    R12 & Город - район & 1:M & 
        Район находится в городе. Следовательно, у города 
        может быть несколько районов, у района - один город \\
    R13 & Район - улица & 1:M & 
        У района может быть несколько улиц. Улица находится в одном районе \\
\end{longtblr}

Все связи представленные в таблице являются безусловными. То есть у каждого 
номера телефона есть АТС, которой он принадлежит. У абонемента всегда
есть абонент и номер телефона. У междугороднего звонка есть инициатор. 
У района - город, в котором он находится. У улицы - район, в котором она 
находится.

\section{Классификация сущностей}

Сущности АТС и Абонент стержневые. 

Номер телефона - характеристичекая сущность, 
так как характеризует АТС к которой будет принадлежать абонент.

Абонемент - ассоциативная сущность, так как характеризует связь M:N между номером и абонентом.

Междугородний звонок - обозначающая сущность.

Сущность АТС является супертипом для городской, 
ведомственной и учрежденческой АТС.

\section{Инфологическая модель}
Описание модели на языке инфологического проектирования:

\textbf{АТС} (идентификатор, название, владелец)

\textbf{Абонент} (идентификатор, фамилия, имя, отчество, пол, 
возраст, льгота, задолжность)

\textbf{Номер телефона} [АТС] (идентификатор, номер, почтовый 
индекс, район, улица, дом, квартира, таксофон?)

\textbf{Абонемент} [номер телефона, абонент] (идентификатор, 
тип телефона, включен?, дата оплаты, межгород включен?, дата оплаты межгорода)

\textbf{Междугородний звонок} [абонемент] (дата и время, длительность, пункт назначения)


\begin{figure}[!ht]
\begin{center}
\includegraphics[scale=0.5]{resources/er_diagram.png}\caption{ER диаграмма}
\end{center}
\end{figure}

\end{document}
