\documentclass{report}

\begin{document}
\chapter{Клиентское приложение}

Приложение написано на языке Java. Для реализации пользовательского
интерфейса использован JavaFX, а для обращения к базе 
данных - PostgreSQL JDBC Driver.

Приложение представляет из себя набор страниц (экранов), 
через которые пользователь может обратиться к БД, 
заполнив форму. Вся логика находится на стороне БД, 
приложение просто отображает полученные данные, минимально
их обрабатывая, т.е. приложение это просто фронтенд БД.

При старте приложение загружает редко обновляемые данные, 
например: серийные номера АТС, названия районов и улиц, 
чтобы упростить выбор пользователю. 
Далее эти данные периодически заново запрашиваются.
Запросы ``Статистика по должникам'' и 
``Популярный город для междугородних звонков'' (4 и 9) 
также автоматически получаются с БД.

\section{Запросы}

Расположение запросов:
\begin{enumerate}
    \item на странице ``Список абонентов''
    \item на странице ``Свободные номера''
    \item на странице ``Список должников''
    \item на странице ``Статистика''
    \item на странице ``Список таксофонов''
    \item на странице ``Соотношение абонентов на АТС'' и ``Соотношение абонентов по районам''
    \item на странице ``Параллельные телефоны''
    \item на странице ``Телефоны по адресу''
    \item на странице ``Статистика''
    \item на странице ``Абоненты по номеру''
    \item на странице ``Необходимо письменное уведомление''
\end{enumerate}

\section{Сценарии использования}

Опишем реализованные сценарии. Реализация в виде
фукнций находится в файле \textbf{citnisdb-usecase.sql}.

\subsection*{Добавить абонента}

Добавляет нового абонента в БД. Реализован в виде обычного INSERT.

\begin{lstlisting}
INSERT INTO subscribers (last_name, first_name, surname, 
    gender, age, benefit)
	VALUES ('Ландау', 'Пелагея', 'Сергеевна', 'f', '25', 0.5);
\end{lstlisting}

\subsection*{Добавить номер телефона}

Добавляет новый номер телефона на указанную АТС.

\begin{lstlisting}
CREATE FUNCTION create_phone_number(ats_serial text, 
    phone_no int, street text, house int, 
    phone_type text) 
RETURNS void AS $$
DECLARE
    ats record;
    address record;
    rec record;
BEGIN
    SELECT ats_id INTO ats 
        FROM ats 
        WHERE serial_no = ats_serial;
    SELECT address_id INTO address
        FROM current_city
        WHERE street_name = street
            AND house_no = house;
    IF NOT FOUND THEN
        SELECT street_id INTO rec 
            FROM streets 
            WHERE street_name = street;
        INSERT INTO addresses (street_id, house_no)
            VALUES(rec.street_id, house);
        SELECT address_id INTO address
            FROM current_city
            WHERE street_name = street
                AND house_no = house;
    END IF;
    INSERT INTO phone_numbers_v (ats_id, phone_no, 
        address_id, phone_type)
        VALUES(ats.ats_id, phone_no, address.address_id, 
            phone_type);
END;
$$ LANGUAGE plpgsql;
\end{lstlisting}

Пример использования:
\begin{lstlisting}
SELECT create_phone_number('7IBFA115116', 5900103,
    'Блюхера', 32, 'Parallel');
\end{lstlisting}

\subsection*{Изменить тип телефона}

Изменяет тип конечного устройства с указанным номером.

\begin{lstlisting}
CREATE FUNCTION change_phone_type(ats_serial text, 
    phone_number int, phone_t text) 
RETURNS void AS $$
DECLARE
    ats record;
    phone record;
BEGIN
    SELECT ats_id INTO ats 
        FROM ats 
        WHERE serial_no = ats_serial;
    SELECT phone_id INTO phone
        FROM phone_numbers_v
        WHERE phone_no = phone_number
            AND ats_id = ats.ats_id;
    UPDATE phone_numbers_v
        SET phone_type = phone_t 
        WHERE phone_id = phone.phone_id;
END;
$$ LANGUAGE plpgsql;
\end{lstlisting}

Пример использования:
\begin{lstlisting}
SELECT change_phone_type('7IBFA115116',
        5900103, 'Common');
\end{lstlisting}

\subsection*{Добавить таксофон}

Добавляет новый таксофон на указанную АТС.

\begin{lstlisting}
CREATE FUNCTION create_payphone(ats_serial text, 
    phone_no int, street text, house int) 
RETURNS void AS $$
DECLARE
    ats record;
    address record;
    rec record;
BEGIN
    SELECT ats_id INTO ats 
        FROM ats 
        WHERE serial_no = ats_serial;
    SELECT address_id INTO address
        FROM current_city
        WHERE street_name = street
            AND house_no = house;
    IF NOT FOUND THEN
        SELECT street_id INTO rec 
            FROM streets 
            WHERE street_name = street;
        INSERT INTO addresses (street_id, house_no)
            VALUES(rec.street_id, house);
        SELECT address_id INTO address
            FROM current_city
            WHERE street_name = street
                AND house_no = house;
    END IF;
    INSERT INTO payphones_v (ats_id, phone_no, address_id)
        VALUES(ats.ats_id, phone_no, address.address_id);
END;
$$ LANGUAGE plpgsql;
\end{lstlisting}

Пример использования:
\begin{lstlisting}
SELECT create_payphone('6IBVG115145',
        1000155, 'Бориса Богаткова', 1);
\end{lstlisting}

\subsection*{Подключить межгород}

Подключает указанному абоненту междугородние звонки.

\begin{lstlisting}
CREATE FUNCTION enable_intercity_calls(ats_serial text, 
    phone_number int, subscriber_last_name text, 
    subscriber_first_name text) 
RETURNS void AS $$
DECLARE
    ats record;
    phone record;
    subscriber record;
    subscription record;
BEGIN
    SELECT ats_id INTO ats 
        FROM ats 
        WHERE serial_no = ats_serial;
    SELECT phone_id INTO phone
        FROM phone_numbers_v
        WHERE phone_no = phone_number
            AND ats_id = ats.ats_id;
    SELECT subscriber_id INTO subscriber
        FROM subscribers
        WHERE last_name = subscriber_last_name
            AND first_name = subscriber_first_name;
    SELECT subscription_id INTO subscription
        FROM subscriptions
        WHERE subscriber_id = subscriber.subscriber_id
            AND phone_id = phone.phone_id;
    INSERT INTO service_connection (subscription_id, 
        service_id)
        VALUES(subscription.subscription_id, 2);
END;
$$ LANGUAGE plpgsql;
\end{lstlisting}

Пример использования:
\begin{lstlisting}
SELECT enable_intercity_calls('7IBFA115116',
        5900103, 'Ландау', 'Пелагея');
\end{lstlisting}

\subsection*{Отключить межгород}

Отключает у указанного абонента междугородние звонки.

\begin{lstlisting}
CREATE FUNCTION disable_intercity_calls(ats_serial text, 
    phone_number int, subscriber_last_name text, 
    subscriber_first_name text) 
RETURNS void AS $$
DECLARE
    ats record;
    phone record;
    subscriber record;
    subscription record;
BEGIN
    SELECT ats_id INTO ats 
        FROM ats 
        WHERE serial_no = ats_serial;
    SELECT phone_id INTO phone
        FROM phone_numbers_v 
        WHERE phone_no = phone_number
            AND ats_id = ats.ats_id;
    SELECT subscriber_id INTO subscriber
        FROM subscribers
        WHERE last_name = subscriber_last_name
            AND first_name = subscriber_first_name;
    SELECT subscription_id INTO subscription
        FROM subscriptions
        WHERE subscriber_id = subscriber.subscriber_id
            AND phone_id = phone.phone_id;
    DELETE FROM service_connection
        WHERE 
            subscription_id = subscription.subscription_id
            AND service_id = 2;
END;
$$ LANGUAGE plpgsql;
\end{lstlisting}

Пример использования:
\begin{lstlisting}
SELECT disable_intercity_calls('7IBFA115116',
    5900103, 'Ландау', 'Пелагея');
\end{lstlisting}

\subsection*{Оформить номер}

Оформляет номер телефона на указанного абонента.

\begin{lstlisting}
CREATE FUNCTION register_phone_number_for_subscriber(
    ats_serial text, phone_number int, 
    sub_last_name text, sub_first_name text, 
    sub_apartment int) 
RETURNS void AS $$
DECLARE
    ats record;
    phone record;
    subscriber record;
BEGIN
    SELECT ats_id INTO ats 
        FROM ats 
        WHERE serial_no = ats_serial;
    SELECT phone_id INTO phone
        FROM phone_numbers_v
        WHERE phone_no = phone_number
            AND ats_id = ats.ats_id;
    SELECT subscriber_id INTO subscriber
        FROM subscribers
        WHERE last_name = sub_last_name
            AND first_name = sub_first_name;
    INSERT INTO subscriptions (phone_id, subscriber_id, 
        apartment)
        VALUES(phone.phone_id, 
            subscriber.subscriber_id, sub_apartment);
END;
$$ LANGUAGE plpgsql;
\end{lstlisting}

Пример использования:
\begin{lstlisting}
SELECT register_phone_number_for_subscriber('7IBFA115116',
    5900103, 'Ландау', 'Пелагея', 4);
\end{lstlisting}

\end{document}
