\documentclass{report}

\begin{document}
\chapter{Введение}

Городская телефонная сеть (ГТС) представляет собой разветвленную сеть 
локальных автоматических телефонных станций (АТС). АТС подразделяются 
на городские, ведомственные и учрежденческие и, возможно, обладают 
характерным только для этой группы набором атрибутов. У каждой АТС есть 
свои абоненты. У абонента может стоять телефон одного из трех типов: 
основной, параллельный или спаренный. За каждым абонентом (у него есть 
фамилия, имя, отчество, пол, возраст и т.д.) закреплен свой номер телефона, 
причем у нескольких абонентов может быть один и тот же номер 
(при параллельном или спаренном телефоне). Каждому номеру телефона 
соответствует адрес (индекс, район, улица, дом, квартира), причем 
параллельные или спаренные телефоны обязательно должны находиться 
в одном доме.

Все телефоны городской АТС имеют выход на межгород, но для конкретного 
абонента он может быть либо открыт, либо закрыт по какой-либо причине 
(отключен по желанию абонента, за неуплату и т.п.). 
Ведомственные и учрежденческие АТС имеют свою внутреннюю замкнутую сеть 
телефонов. Сведения о междугородных переговорах 
собираются и анализируются на ГТС.

Абоненты обязаны платить абонентскую плату. Плата должна вноситься каждый 
месяц до 20-го числа. При неуплате после письменного уведомления в течение 
двух суток отключается абонент. При задолженности за междугородние 
разговоры и неоплате после письменного уведомления производится отключение 
только возможности выхода на межгород. Включение того и (или) другого 
производится при оплате стоимости включения, абонентской платы и пени.

Абонентов любой АТС можно подразделить на простых и льготных. К категории 
льготников относятся пенсионеры, инвалиды и т.д. Льготники платят только 
50\% абонентской платы. В соответствии со всем этим (тип телефона, льготник 
или нет, есть ли выход на межгород) рассчитывается размер абонентской платы.

На установку телефона существуют очереди: льготная и обычная. При подходе 
очередности рассматривается техническая возможность установки (наличие 
кабеля и свободного канала, наличие свободных телефонных номеров).

В городе также существуют общественные телефоны и таксофоны, расположенные 
по определенным адресам.

\end{document}
